% !TeX root = main.tex


\section{Algebra of Rational
Expressions}


\subsection{Simplified rational expressions}
Let \(p(x)\) and \(q(x)\) be polynomials in on variable \(x\) and \(q\) is not a
constant function. We call the quotient \(r(x)=\frac{p(x)}{q(x)}\) a
\textbf{\emph{rational expression}}. A rational expression is
\textbf{\emph{simplified}} if the numerator and the denominator have no
common factor other than \(1\).

Two rational expressions are \textbf{\emph{equivalent}} if they simplify into the same rational expression.

Let \(p(x)\), \(q(x)\) be polynomials with \(q(x)\neq 0\) and \(c(x)\)
be a nonzero expression. Then \[
\frac{~p(x)\cdot c(x)~}{~q(x)\cdot c(x)~}=\frac{~p(x)~}{~q(x)~}.
\]

\begin{example}
  Simplify \(\dfrac{x^2+4x+3}{x^2+3x+2}\).
\end{example}
\vspace*{4\baselineskip}

\begin{example}
  Simplify \(\dfrac{2x^2-x-3}{2x^2-3x-5}\).
\end{example}
\vspace*{4\baselineskip}

\hypertarget{multiplying-rational-expressions}{%
\subsection{Multiplying Rational
Expressions}\label{multiplying-rational-expressions}}

If \(p(x)\), \(q(x)\), \(s(x)\), \(t(x)\) are polynomials with \(q(x)\neq 0\) and
\(t(x)\neq 0\), then \[
\frac{~p(x)~}{~q(x)~}\cdot\frac{~s(x)~}{~t(x)~}=\frac{~p(x)s(s)~}{~q(x)t(x)~}.
\]

\begin{example}
  Multiply and then simplify.
  \[\frac{3x^2}{x^2+x-6}\cdot\frac{x^2-4}{6x}.\]
\end{example}
\vspace*{4\baselineskip}

\begin{example}
  Multiply and then simplify. \[
  \frac{3x^2-8x-3}{x^2+8x+16}\cdot\frac{x^2-16}{5x^2-14x-3}.
  \]
\end{example}
\vspace*{4\baselineskip}


\hypertarget{dividing-rational-expressions}{%
\subsection{Dividing Rational
Expressions}\label{dividing-rational-expressions}}

If \(p(x)\), \(q(x)\), \(s(x)\), \(t(x)\) are polynomials where \(q(x)\neq 0\),
\(s(x)\neq 0\) and \(t(x)\neq 0\), then \[
\frac{~p(x)~}{~q(x)~}\div\frac{~s(x)~}{~t(x)~}=\frac{~p(x)~}{~q(x)~}\cdot\frac{~t(x)~}{~s(x)~}=\frac{~p(x)t(x)~}{~q(x)s(x)~}.
\]

\begin{example}
  Divide and then simplify.\\
  \[
  \frac{2x+6}{x^2-6x-7}\div \frac{6x-2}{2x^2-x-3}.
  \]
\end{example}
\vspace*{4\baselineskip}

\hypertarget{adding-or-subtracting-rational-expressions-with-the-same-denominator}{%
\subsection{Adding or Subtracting Rational Expressions with the Same
Denominator}\label{adding-or-subtracting-rational-expressions-with-the-same-denominator}}

If \(p(x)\), \(q(x)\) and \(r(x)\) are polynomials with \(r(x)\neq 0\), then \[
\frac{~p(x)~}{~r(x)~}+\frac{~q(x)~}{~r(x)~}=\frac{~p(x)+q(x)~}{~r(x)~}\qquad \text{and}\]
\[
\frac{~p(x)~}{~r(x)~}-\frac{~q(x)~}{~r(x)~}=\frac{~p(x)-q(x)~}{~r(x)~}.
\]

\begin{example}
  Add and simplify\\
  \[
  \frac{x^2+4}{x^2+3x+2}+\frac{5x}{x^2+3x+2}.
  \]
\end{example}
\vspace*{4\baselineskip}

\begin{example}
  Subtract and simplify \(\dfrac{2x^2}{2x^2-x-3}-\dfrac{3x+5}{2x^2-x-3}\).
\end{example}
\vspace*{4\baselineskip}

\hypertarget{adding-or-subtracting-rational-expressions-with-different-denominators}{%
\subsection{Adding or Subtracting Rational Expressions with Different
Denominators}\label{adding-or-subtracting-rational-expressions-with-different-denominators}}

To add or subtract rational expressions with different denominators, we
need to rewrite the rational expressions to equivalent rational
expressions with the same denominator, say the LCD.

\begin{example}
  Find the LCD of \(\dfrac{3}{x^2-x-6}\) and \(\dfrac{6}{x^2-4}\).
\end{example}
\vspace*{4\baselineskip}

\begin{example}
  Subtract and simplify\\
  \[\frac{x-3}{x^2-2x-8}- \frac{1}{x^2-4}\]
\end{example}
\vspace*{4\baselineskip}

\hypertarget{simplifying-complex-rational-expressions}{%
\subsection{Simplifying Complex Rational
Expressions}\label{simplifying-complex-rational-expressions}}

A \textbf{\emph{complex rational expression}} is a rational expression
whose denominator or numerator contains a rational expression.

\begin{example}
  Simplify \[
  \frac{~\frac{2x-1}{x^2-1}+\frac{x-1}{x+1}~}{~\frac{x+1}{x-1}-\frac{1}{x^2-1}~}
  \]
\end{example}
\vspace*{4\baselineskip}

\subsection{Practice}

\begin{exercise}
  Simplify.
  
  \begin{enumerate}
  \item
    \(\dfrac{3x^2-x-4}{x+1}\)
  \item
    \(\dfrac{2x^2-x-3}{2x^2+3x+1}\)
  \item
    \(\dfrac{x^2-9}{3x^2-8x-3}\)
  \end{enumerate}
\end{exercise}

\begin{exercise}
  Multiply and simplify.
  
  \begin{enumerate}
  \item
    \(\dfrac{x+5}{x+4}\cdot\dfrac{x^2+3x-4}{x^2-25}\)
  \item
    \(\dfrac{3x^2-2x}{x+2}\cdot\dfrac{3x^2-4x-4}{9x^2-4}\)
  \item
    \(\dfrac{6y-2}{3-y}\cdot\dfrac{y^2-6y+9}{3y^2-y}\)
  \end{enumerate}
\end{exercise}

\begin{exercise}
  Divide and simplify.
  
  \begin{enumerate}
  \item
    \(\dfrac{9x^2-49}{6}\div\dfrac{3x^2-x-14}{2x+4}\)
  \item
    \(\dfrac{x^2+3x-10}{2x-2}\div\dfrac{x^2-5x+6}{x^2-4x+3}\)
  \item
    \(\dfrac{y-x}{xy}\div\dfrac{x^2-y^2}{y^2}\)
  \end{enumerate}
\end{exercise}

\begin{exercise}
  Simplify. \[
  \frac{-x^2+11x-18}{x^2-4x+4}\div \frac{x^2-5x-36}{x^2-7x+12}\cdot \frac{2x^2+7x-4}{x^2+2x-15}
  \]
\end{exercise}
\vspace*{4\baselineskip}

\begin{exercise}
  Add/subtract and simplify.
  
  \begin{enumerate}
  \item
    \(\dfrac{x^2+2x-2}{x^2+2x-15}+\dfrac{5x+12}{x^2+2x-15}\)
  \item
    \(\dfrac{3x-10}{x^2-25}-\dfrac{2x-15}{x^2-25}\)
  \item
    \(\dfrac{4}{(x-3)(x+2)}+\dfrac{3x-2}{x^2-x-6}\)
  \end{enumerate}
\end{exercise}

\begin{exercise}
  Find the LCD of rational expressions.
  
  \begin{enumerate}
  \item
    \(\dfrac{2x}{2x^2-5x-3}\)\quad and \quad \(\dfrac{x-1}{x^2-x-6}\)
  \item
    \(\dfrac{9}{7x^2-28x}\) \quad and \quad \(\dfrac{2}{x^2-8x+16}\)
  \end{enumerate}
\end{exercise}

\begin{exercise}
  Add and simplify.
  
  \begin{enumerate}
  \item
    \(\dfrac{x}{x+1}+\dfrac{x-1}{x+2}\)
  \item
    \(\dfrac{x+2}{2x^2-x-3}+\dfrac{1}{x^2+3x+2}\)
  \item
    \(\dfrac{4}{x-3}+\dfrac{3x-2}{x^2-x-6}\)
  \end{enumerate}
\end{exercise}

\begin{exercise}
  Subtract and simplify.
  
  \begin{enumerate}
  \item
    \(\dfrac{3x+5}{x^2-7x+12}-\dfrac{3}{x-3}\)
  \item
    \(\dfrac{y}{y^2-5y-6}-\dfrac{7}{y^2-4y-5}\)
  \item
    \(\dfrac{2x-3}{x^2+3x-10}-\dfrac{x+2}{x^2+2x-8}\)
  \end{enumerate}
\end{exercise}

\begin{exercise}
  Simplify. \[
  \dfrac{x+11}{7x^2-2x-5}+\dfrac{x-2}{x-1}-\dfrac{x}{7x+5}
  \]
\end{exercise}
\vspace*{4\baselineskip}

\begin{exercise}
  Subtract and simplify. \[
  \dfrac{x-1}{x^2-3x}+\dfrac{4}{x^2-2\:x-3}-\dfrac{1}{x\left(x+1\right)}
  \]
\end{exercise}
\vspace*{4\baselineskip}

\begin{exercise}
  Simplify.
  
  \begin{enumerate}
  \item
    \(\dfrac{~1+\dfrac{2}{x}~}{~1-\dfrac{2}{x}~}\)
  \item
    \(\dfrac{~\dfrac{1}{x^2}-1~}{~\dfrac{1}{x^2}-\dfrac{1}{x}~}\)
  \end{enumerate}
\end{exercise}

\begin{exercise}
  Simplify.
  
  \begin{enumerate}
  \item
    \(\dfrac{~\dfrac{x^2-y^2}{y^2}~}{~\dfrac1x-\dfrac{1}{y}~}\)
  \item
    \(\dfrac{~\dfrac{2}{(x+1)^2}-\dfrac{1}{x+1}~}{~1-\dfrac{4}{(x+1)^2}~}\)
    \item
      \(\dfrac{~\dfrac{5x}{x^2-x-6}~}{~\dfrac2{x+2}+\dfrac{3}{x-3}~}\)
    \item
      \(\dfrac{~\dfrac{x+1}{x-1}+\dfrac{x-1}{x+1}~}{~\dfrac{x+1}{x-1}-\dfrac{x-1}{x+1}~}\)
  \end{enumerate}
\end{exercise}

\begin{exercise}
  Tim and Jim refill their cars at the same gas station twice last month.
  Each time Tim got \$20 gas and Jim got 8 gallon. Suppose they refill
  their cars on same days. The price was \$2.5 per gallon the first time.
  The price on the second time changed. Can you find out who had the
  better average price?
\end{exercise}
\vspace*{4\baselineskip}

