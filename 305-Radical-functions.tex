% !TeX root = main.tex

\hypertarget{radical-functions}{%
\section{Radical Functions}\label{radical-functions}}

\hypertarget{the-domain-of-a-radical-function}{%
\subsection{The Domain of a Radical
Function}\label{the-domain-of-a-radical-function}}

A \textbf{\emph{radical function}} \(f\) is defined by an equation
\(f(x)=\sqrt[n]{r(x)}\), where \(r(x)\) is an algebraic expression. For
example \(f(x)=\sqrt{x+1}\). When \(n\) is odd number, \(r(x)\) can be
any real number. When \(n\) is even, \(r(x)\) has to be nonnegative,
that is \(r(x)\geq 0\) so that \(f(x)\) is a real number.

\begin{example}

Find the domain of the function \(f(x)=\sqrt{x+1}\).

\end{example}
\vspace*{6\baselineskip}

\subsection{Practice}

\begin{exercise}

Find the domain of each function. Write in interval notation.

\begin{enumerate}
\item
  \(f(x)=\sqrt{1-x^2}\)
\item
  \(f(x)=-\sqrt{\frac{1}{x-5}}\)
\end{enumerate}

\end{exercise}
