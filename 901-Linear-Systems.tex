% !TeX root = main.tex

\section{Methods to Solve a Linear
Systems}

A \textbf{\emph{system of linear equations}} of two variables consists
of two equations. A \textbf{\emph{solution of a system}} of linear
equations of two variables is an ordered pair that satisfies both
equations.

\hypertarget{substitution-method}{%
\subsection{Substitution Method}\label{substitution-method}}

\begin{example}

Solve the system of linear equations using the substitution method. \[
\begin{aligned}
    x + y =  & 3  \label{ex-1-1}\\
    2x + y = & 4
\end{aligned}
\]

\end{example}
\vspace*{6\baselineskip}

\hypertarget{elimination-method}{%
\subsection{Elimination Method}\label{elimination-method}}

\begin{example}

Solve the system of linear equations using the addition method. \[
\begin{aligned}
5x + 2y & = 7  \label{ex-2-1} \\
3x - y  & = 13
\end{aligned}
\]

\end{example}
\vspace*{6\baselineskip}

\begin{remark}

A linear system may have \textbf{\emph{one solution}}, \textbf{\emph{no
solution}} or \textbf{\emph{infinitely many solutions}}.

If the lines defined by equations in the linear system have the same
slope but different \(y\)-intercepts or the elimination method ends up
with \(0=c\), where \(c\) is a nonzero constant, then the system has no
solution.

If the lines defined by equations in the linear system have the same
slope and the same \(y\)-intercept or the elimination method ends up
with \(0=0\), then the system has infinitely many solutions. In this
case, we say that the system is \textbf{\emph{dependent}} and a solution
can be expressed in the form \((x, f(x))=(x, mx+b)\).

\end{remark}

\subsection{Practice}

\begin{exercise}

Solve. \[
\begin{aligned}
    2x-y   & =8 \\
    -3x-5y & =1
\end{aligned}
\]

\end{exercise}
\vspace*{6\baselineskip}

\begin{exercise}

Solve. \[
\begin{aligned}
    x+4y=  & 10  \\
    3x-2y= & -12
\end{aligned}
\]

\end{exercise}
\vspace*{6\baselineskip}

\begin{exercise}

Solve. \[
\begin{aligned}
    -x-5y & =29  \\
    7x+3y & =-43
\end{aligned}
\]

\end{exercise}
\vspace*{6\baselineskip}

\begin{exercise}

Solve. \[
\begin{aligned}
    2x+15y= & 9   \\
    x-18y=  & -21
\end{aligned}
\]

\end{exercise}
\vspace*{6\baselineskip}

\begin{exercise}

Solve. \[
\begin{aligned}
    2x+7y & =5  \\
    3x+2y & =16
\end{aligned}
\]

\end{exercise}
\vspace*{6\baselineskip}

\begin{exercise}

Solve. \[
\begin{aligned}
    4x+3y=  & -10 \\
    -2x+5y= & 18
\end{aligned}
\]

\end{exercise}
\vspace*{6\baselineskip}

\begin{exercise}

Solve. \[
\begin{aligned}
    3x+2y & =6  \\
    6x+4y & =16
\end{aligned}
\]

\end{exercise}
\vspace*{6\baselineskip}

\begin{exercise}

Solve. \[
\begin{aligned}
    2x-3y  & =-6 \\
    -4x+6y & =12
\end{aligned}
\]

\end{exercise}
\vspace*{6\baselineskip}

\begin{exercise}

Last week, Mike got 5 apples and 4 oranges for \$7. The week the prices
are still the same and he got 3 apples and 6 oranges for \$6. What's the
price for 1 apple and 1 orange?

\end{exercise}
\vspace*{6\baselineskip}

\begin{exercise}

The sum of the digits of a certain two-digit number is 7. Reversing its
digits increases the number by 27.\\
What is the number?

\end{exercise}

