% !TeX root = main.tex

\hypertarget{completing-the-square}{%
\section{Completing the Square}\label{completing-the-square}}

\textbf{The square root property}:\\
Suppose that \(X^2=d\). Then \(X=\sqrt{d}\) or \(X=-\sqrt{d}\), or
simply \(X=\pm\sqrt{d}\).

The square root property provides another method to solve a quadratic
equation, completing the square. 

Let \(h=-\frac{b}{2a}\) and  \(k=ah^2+bh+c\).
Then \[
ax^2+bx+c=a(x-h)^2+k=a\left(x+\frac{b}{2a}\right)^2+\frac{4ac-b^2}{4a^2}.
\]

The procedure to rewrite a trinomial as the sum of a perfect square and
a constant is called \textbf{\emph{completing the square}}.

\begin{example}

Solve the equation \(x^2+2x-1=0\).

\end{example}
\vspace*{5\baselineskip}

\begin{example}

Solve the equation \(-2x^2+8x-9=0\).

\end{example}
\vspace*{5\baselineskip}

\hypertarget{the-quadratic-formula}{%
\subsection{The Quadratic Formula}\label{the-quadratic-formula}}

Using the method of completing the square, we obtain the follow
quadratic formula for the quadratic equation \(ax^2+bx+c=0\) with
\(a\neq 0\): \[
    x=\frac{-b\pm\sqrt{b^2-4ac}}{2a}.
\]

The quantity \(b^2-4ac\) is called the \textbf{\emph{discriminant}} of
the quadratic equation.

\begin{enumerate}[sepno]
\item
  If \(b^2-4ac>0\), the equation has two real solutions.
\item
  If \(b^2-4ac=0\), the equation has one real solution.
\item
  If \(b^2-4ac<0\), the equation has two imaginary solutions (no real
  solutions).
\end{enumerate}

\begin{example}

Determine the type and the number of solutions of the equation
\((x-1)(x+2)=-3\).

\end{example}
\vspace*{5\baselineskip}

\begin{example}

Solve the equation \(2x^2-4x+7=0\).

\end{example}
\vspace*{5\baselineskip}

\begin{example}

Find the base and the height of a \textbf{triangle} whose base is three
inches more than twice its height and whose area is \(5\) square inches.
Round your answer to the nearest tenth of an inch.

\end{example}
\vspace*{5\baselineskip}

\hypertarget{practice}{%
\subsection{Practice}\label{practice}}

\begin{exercise}

Solve the quadratic equation by the square root property.

\begin{enumerate}
\item
  \(2x^2-6=0\)
\item
  \((x-3)^2=10\)
\item
  \(4(x+1)^2+25=0\)
\end{enumerate}

\end{exercise}

\begin{exercise}

Solve the quadratic equation by completing the square.

\begin{enumerate}
\item
  \(x^2+x-1=0\)
\item
  \(x^2+8x+12=0\)
\item
  \(3x^2+6x-1=0\)
\end{enumerate}

\end{exercise}

\begin{exercise}

Determine the number and the type of solutions of the given equation.

\begin{enumerate}
\item
  \(x^2+8x+3=0\)
\item
  \(3x^2-2x+4=0\)
\item
  \(2x^2-4x+2=0\)
\end{enumerate}

\end{exercise}

\begin{exercise}

Solve using the quadratic formula.

\begin{enumerate}
\item
  \(x^2+3x-7=0\)
\item
  \(2x^2=-4x+5\)
\item
  \(2x^2=x-3\)
\end{enumerate}

\end{exercise}

\begin{exercise}

Solve using the quadratic formula.

\begin{enumerate}
\item
  \((x-1)(x+2)=3\)
\item
  \(2x^2-x=(x+2)(x-2)\)
\item
  \(\frac12 x^2+x= \frac13\)
\end{enumerate}

\end{exercise}

\begin{exercise}

A \textbf{triangle} whose area is \(7.5\) square meters has a base that
is one meter less than triple the height. Find the length of its base
and height. Round to the nearest hundredth of a meter.

\end{exercise}
\vspace*{5\baselineskip}

\begin{exercise}

A \textbf{rectangular} garden whose length is \(2\) feet longer than its
width has an area 66 square feet. Find the dimensions of the garden,
rounded to the nearest hundredth of a foot.

\end{exercise}

