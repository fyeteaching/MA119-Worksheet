% !TeX root = main.tex


\section{Applications of Exp and Log Functions}

\hypertarget{newtons-law-of-cooling}{%
\subsection{Newton's Law of Cooling}\label{newtons-law-of-cooling}}

Suppose an object with an initial temperature \(T(0)\) is placed in an
environment with surrounding temperature \(T_{\text{env}}\). By
\href{https://en.wikipedia.org/wiki/Newton\%27s_law_of_cooling}{Newton's
Law of Cooling}, after \(t\) minutes, the temperature of the object
\(T(t)\) is given by the exponential function \[
T(t)=T_{\text{env}}+(T(0)-T_{\text{env}})\,e^{-rt},
\] where \(r\) is a positive constant characteristic of the system.

A cup of coffee is brewed with a temperature 195°F and placed in a room
with the temperature 60°F. The cooling constant for a cup of coffee is $r = 0.09 \text{min}^{-1}$.

\begin{enumerate}
\item
  After 30 minutes, what is the temperature of the coffee?
\item
  How long it takes for the coffee to cool down to the room temperature?
\end{enumerate}


\subsection{Exponential and Logarithmic Equations}

To solve an exponential or logarithmic equation, the first step is to rewrite the equation with a single exponentiation or logarithm.
Then we can use the equivalent relation between exponentiation and logarithm to rewrite the equation and solve the resulting equation.

\begin{example}

Solve the equation \(10^{2x-1}-5=0\).

\end{example}
\vspace*{6\baselineskip}

\begin{example}

Solve the equation \(\log_2 x + \log_2 (x - 2) = 3\).

\end{example}
\vspace*{6\baselineskip}

\subsection{Solving Compound Interest
Model}

\begin{example}

A check of \$5000 was deposited in a savings account with an annual
interest rate \(6\%\) which is compounded monthly. How many years will
it take for the money to raise by 20\%?

\end{example}
\vspace*{6\baselineskip}

\subsection{Practice}

\begin{exercise}

Solve the exponential equation.

\begin{enumerate}
\item
  \(2^{x-1}=4\)
\item
  \(7e^{2x}-5=58\)
\end{enumerate}

\end{exercise}

\begin{exercise}

Solve the exponential equation.

\begin{enumerate}
\item
  \(3^{x^2-2x}=e^{-\ln3}\)
\item
  \(2^{(x+1)}=3^{(1-x)}\)
\end{enumerate}

\end{exercise}

\begin{exercise}

Solve the logarithmic equation.

\begin{enumerate}
\item
  \(\log_5x+\log_5(4x-1)=1\)
\item
  \(\ln \sqrt{x+1}=1\)
\end{enumerate}

\end{exercise}

\begin{exercise}

Solve the logarithmic equation.

\begin{enumerate}
\item
  \(\log_2(x+2)-\log_2(x-5)=3\)
\item
  \(\log_3(x-5)=2-\log_3(x+3)\)
\end{enumerate}

\end{exercise}

\begin{exercise}

For the given function, find values of \(x\) satisfying the given
equation.

\begin{enumerate}
\item
  \(f(x)=\log_4x-2\log_4(x+1)\),\quad \(f(x)=-1\)
\item
  \(g(x)=\log(2-5x)+\log(-x)\),\quad \(g(x)=1\)
\end{enumerate}

\end{exercise}

\begin{exercise}

Find intersections of the given pairs of curves.

\begin{enumerate}
\item
  \(f(x)=e^{x^2}\) and \(g(x)=e^x+12\).
\item
  \(f(x)=\log_7\left(\frac12(x+2)\right)\) and
  \(g(x)=1-\log_7(x-3)\)
\end{enumerate}

\end{exercise}

\begin{exercise}

Using the formula \(A=P(1+\frac rn)^{nt}\) to determine how many years,
to the nearest hundredth, it will take to double an investment \$20,000
at the interest rate 5\% compounded monthly.

\end{exercise}
\vspace*{6\baselineskip}

\begin{exercise}

Newton's Law of Cooling states that the temperature \(T\) of an object
at any time \(t\) satisfying the equation \(T=T_s+(T_0-T_s)e^{-rt}\),
where \(T_s\) is the the temperature of the surrounding environment,
\(T_0\) is the initial temperature of the object, and \(r\) is positive
constant characteristic of the system, which is in units of
\({\displaystyle \text{time}^{-1}}\). In a room with a temperature of
\(22 ^\circ C\), a cup of tea of \(97 ^\circ C\) was freshly brewed.
Suppose that \(r=\ln 5/20~~\text{minute}^{-1}\). In how many minutes,
the temperature of the tea will be \(37 ^\circ C\)?

\end{exercise}

