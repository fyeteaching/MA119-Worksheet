\usepackage{geometry}
\geometry{
    letterpaper,
    left=0.8in,
    top=0.8in,
    headsep=\baselineskip,
    textwidth=26pc,
    marginparsep=2pc,
    marginparwidth=12pc,
    textheight=38\baselineskip,
    headheight=2\baselineskip,
    includemp,
    reversemarginpar,
    bindingoffset=1cm,
    twoside,
    asymmetric
}

\usepackage{amsmath, amsthm}

\usepackage{libertine}  %%%%%The Linux Libertine font family
\usepackage[libertine]{newtxmath}

\usepackage{datetime}
\newdateformat{mydate}{\monthname[\THEMONTH], \THEYEAR}

\newdateformat{lastupdated}
{\monthname[\THEMONTH] \THEDAY, \THEYEAR}

\newdateformat{semester}{
  \ifthenelse{\THEMONTH=1}{Winter \THEYEAR}{
    \ifthenelse{\THEMONTH<6}{Spring \THEYEAR}{
      \ifthenelse{\THEMONTH>8}{Fall \THEYEAR}{
        Summer \THEYEAR
      }
    }
  }
}

\usepackage{titling}

\renewcommand{\sectionmark}[1]{ \markright{#1}{} }

\usepackage{fancyhdr}
\pagestyle{fancy}

\fancyhf{}
\fancyhfoffset[L]{14pc}

\renewcommand{\headrulewidth}{1pt}
\renewcommand{\footrulewidth}{1pt}
\fancyhead[RE,LO]{\bf \course}
% \fancyhead[RO,LE]{\thepage}
\fancyhead[C]{\bf \leftmark}
\fancyfoot[RE, LO]{\raisebox{-10\baselineskip}{\semester{\thedate}}}
\fancyfoot[RO, LE]{\raisebox{-10\baselineskip}{\hfill \sffamily \theauthor}\hspace{-1ex}}
\fancyfoot[C]{\raisebox{-10\baselineskip}{\thepage}}

\usepackage{ifthen}
\usepackage{xparse}
\usepackage{ifoddpage}

\usepackage{eso-pic}

\NewDocumentCommand{\addBG}{}{
  \AddToShipoutPicture{
    \AtTextLowerLeft{
      \put(-1pc,\LenToUnit{-\baselineskip}){
        \rule[0em]{1.5pt}{\dimexpr \textheight+2\baselineskip}
      }
    }
  }
}

\NewDocumentCommand{\newlecture}{}{
  \newpage
  \checkoddpage
  \ifoddpage
  \else
    \clearpage
    \thispagestyle{empty}
    \ClearShipoutPictureBG
    \cleardoublepage
    \newpage
    \addBG
  \fi
}

\usepackage[
  breaklinks = true,
  colorlinks = true,
  pdftitle = "College Algebra Worksheets",
  pdfauthor = "Dr. Ye"
]{hyperref}
\usepackage{bookmark}
\usepackage{longtable}
\usepackage{calc}
\usepackage{booktabs}
\usepackage{array}
\usepackage{multirow}
\usepackage{multicol}
\usepackage{float}
\usepackage[normalem]{ulem}
\usepackage{makecell}
\usepackage{xcolor}

\usepackage{changepage}

\newenvironment{fullwidth}{%
  \begin{adjustwidth}{-14pc}{}%
  \hsize=\linewidth%
}{
  \end{adjustwidth}
}

\usepackage[inline]{enumitem}
\setenumerate{
	label=\textup{(\arabic*)},
	% afterlabel={\quad},
	%%vertical
	topsep=0pt,
	partopsep=0pt,
	itemsep=5\baselineskip,
	parsep=2pt,
  after=\vspace*{\dimexpr 5\baselineskip},
	% labelindent=0em,
	% itemindent = *,
	itemindent=1ex,
	wide,
	itemjoin={\hspace{\fill}},
	%%Horizontal
}
\setitemize{
	%%vertical
	topsep=0pt,
	partopsep=0pt,
	itemsep=0pt,
	parsep=0pt,
	%%Horizontal
	labelindent=0em,
	leftmargin =!,
	itemindent = 0pt,
	labelsep= 2pt,
	labelwidth=1em,
}
\setlist{topsep=0pt}

\SetEnumitemKey{sepno}{nosep, after=\vspace*{0pt}}

\SetEnumitemKey{twocol}{
itemsep = 1\itemsep,
parsep = 1\parsep,
before = \raggedcolumns\begin{multicols}{2},
after = \end{multicols}}

\usepackage{tikz}
\usepackage{pgfplots}
\pgfplotsset{compat=newest}
\usepackage{pgfmath}
\usepackage{tikz-cd}

%%%%%%%%%%%%%%% include files/Figure %%%%%%%%%%%%%%%%%%%%%%%%%%%%%%%%%
\usepackage{import}
% \usepackage{subfiles}
\usepackage[verbose]{wrapfig}
%%%%%%%%%%%%%%%%%%%%%%%%%%%%%%%%%%%%%%%%%%%%%%%%%%%%%%%%%%%%%%%%%

%%%%%%%%%%%%%%%% Cancel common factors in Math %%%%%%%%%%%%%%%%%%%%
\usepackage[makeroom]{cancel}
%%%%%%%%%%%%%%%%%%%%%%%%%%%%%%%%%%%%%%%%%%%%%%%%%%%%%%%%%%%%%%%%%%%

%%%%%%%%%%%%%% Math mode without vertical spacing %%%%%%%%%%%%%%%%%
\makeatletter
\g@addto@macro\normalsize{%
	\setlength\abovedisplayskip{1pt plus 2pt minus 2pt}%
	\setlength\belowdisplayskip{1pt plus 2pt minus 2pt}%
	\setlength\abovedisplayshortskip{1pt plus 2pt minus 2pt}%
	\setlength\belowdisplayshortskip{1pt plus 2pt minus 2pt}%
}
\makeatother
%%%%%%%%%%%%%%%%%%%%%%%%%%%%%%%%%%%%%%%%%%%%%%%%%%%%%%%%%%%%%%%%

\newcommand{\ZZ}{\mathbf{Z}}
\newcommand{\RR}{\mathbf{R}}
\newcommand{\NN}{\mathbf{N}}
\newcommand{\QQ}{\mathbf{Q}}
\newcommand{\abs}[1]{\lvert #1\rvert}
\newcommand{\ii}{\mathbf{i}}
\newcommand{\parll}{ {\mathbin{\parallel}} }
\newcommand{\prll}{{\mathbin{\!/\mkern-5mu/\!}}}

\renewcommand{\bar}{\widebar}
\newcommand*\centermath[1]{\omit\hfil~$\displaystyle#1$~\hfil\ignorespaces}
\newcommand{\cmc}{\centermath}
\newcommand*\ctc[1]{\omit\hfil\quad~ #1 ~\quad\hfil\ignorespaces}
\newcommand{\dfn}[1]{\textit{\textbf{#1}}}


\theoremstyle{plain}% default
\newtheorem{theorem}{Theorem}[section]
\newtheorem{lemma}[theorem]{Lemma}
\newtheorem{proposition}[theorem]{Proposition}
\newtheorem{corollary}[theorem]{Corollary}
\theoremstyle{definition}
\newtheorem{definition}[theorem]{Definition}
\newtheorem{example}{Example}[section]
\newtheorem{exercise}{Exercise}[section]
\theoremstyle{remark}
\newtheorem*{remark}{Remark}
\newtheorem*{note}{Note}
\newtheorem{case}{Case}
