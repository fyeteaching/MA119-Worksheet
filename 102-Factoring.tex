% !TeX root = main.tex

\hypertarget{review-of-factoring}{%
\section{Review of Factoring}\label{review-of-factoring}}

% \hypertarget{can-you-beat-a-calculator}{%
% \subsection{Can You Beat a Calculator}\label{can-you-beat-a-calculator}}

% Do you know a faster way to find the values?

% \begin{enumerate}
% \tightlist
% \item
%   Find the value of the polynomial \(2x^3-98x\) when \(x=-7\).
% \item
%   Find the value of the polynomial \(x^2-9x-22\) when \(x=11\).
% \item
%   Find the value of the polynomial \(x^3-2x^2-9x+18\) when \(x=-3\).
% \item
%   Find the value of \(16^2-14^2\).
% \end{enumerate}

\hypertarget{factor-by-removing-the-gcf}{%
\subsection{Factor by Removing the
GCF}\label{factor-by-removing-the-gcf}}

\textbf{\emph{The greatest common factor (GCF)}} of two polynomials is a
polynomial that all common factors divide it.

To \textbf{\emph{factor a polynomial}} is to \textbf{express the
polynomial as a product} of polynomials of lower degrees.

\begin{example}
  Factor \(4x^3y-8x^2y^2+12x^3y^3\).
\end{example}
\vspace*{4\baselineskip}

\hypertarget{factor-by-grouping}{%
\subsection{Factor by Grouping}\label{factor-by-grouping}}

\begin{example}
  Factor \(2x^2-6xy+xz-3yz\).
\end{example}
\vspace*{4\baselineskip}

\begin{example}
  Factor \(ax+4b-2a-2bx\).
\end{example}
\vspace*{4\baselineskip}

\hypertarget{factor-difference-of-powers}{%
\subsection{Factor Difference of
Powers}\label{factor-difference-of-powers}}

\[a^n-b^n=(a-b)(a^{n-1}+a^{n-2}b+\cdots +ab^{n-2}+b^{n-1})\]

In particular,

\[a^2-b^2=(a-b)(a+b).\]

\begin{example}
  Factor \(25x^2-16\).
\end{example}
\vspace*{4\baselineskip}

\begin{example}
  Factor \(32x^3y-2xy^5\) completely.
\end{example}
\vspace*{4\baselineskip}

\hypertarget{factor-trinomials}{%
\subsection{Factor Trinomials}\label{factor-trinomials}}

If a trinomial \(ax^2+bx+c\), \(A\neq 0\), can be factored, then it can
be expressed as a product of two binomials:\\
\[ax^2+bx+c=(mx+n)(px+q),\] where $m$, $n$, $p$ and $q$ satisfies the following equations.
\[
a=\underbrace{mn}_{\mathrm{F}}\quad\quad\quad
b=\underbrace{mq}_{\mathrm{O}}~\underset{+}{\underset{}{+}}~\underbrace{np}_{\mathrm{I}}
\quad\quad\quad
c=\underbrace{nq}_{\mathrm{F}}
\]

A trinomial \(ax^2+bx+c\) is also called a \textbf{\emph{quadratic
polynomial}}.

\begin{example}
  Factor \(x^2+6x+8\).
\end{example}
\vspace*{4\baselineskip}

\begin{example}
  Factor \(2x^2+5x-3\).
\end{example}
\vspace*{4\baselineskip}

\begin{example}
  Factor the trinomial completely.
  \[4x^4-x^2-3\]
\end{example}
\vspace*{4\baselineskip}

\subsection{Practice}

\begin{exercise}
  Factor out the GCF.
  
  \begin{enumerate}
  \item
    \(18x^2y^2-12xy^3-6x^3y^4\)
  \item
    \(5x(x-7)+3y(x-7)\)
  \item
    \(-2a^2(x+y)+3a(x+y)\)
  \end{enumerate}
\end{exercise}

\begin{exercise}
  Factor by grouping.
  
  \begin{enumerate}
  \item
    \(12xy-10y+18x-15\)
  \item
    \(12ac-18bc-10ad+15bd\)
  \item
    \(5ax-4bx-5ay+4by\)
  \end{enumerate}
\end{exercise}

\begin{exercise}
  Factor completely.
  
  \begin{enumerate}
  \item
    \(25x^2-4\)
  \item
    \(8x^3-2x\)
  \item
    \(25xy^2+x\)
  \end{enumerate}
\end{exercise}

\begin{exercise}
  Factor completely.
  
  \begin{enumerate}
  \item
    \(3x^3+6x^2-12x-24\)
  \item
    \(x^4+3x^3-4x^2-12x\)
  \end{enumerate}
\end{exercise}

\begin{exercise}
  Factor the trinomial.
  
  \begin{enumerate}
  \item
    \(x^2+4x+3\)
  \item
    \(x^2+6x-7\)
  \item
    \(x^2-3x-10\)
  \end{enumerate}
\end{exercise}

\begin{exercise}
  Factor the trinomial.
  
  \begin{enumerate}
  \item
    \(5x^2+7x+2\)
  \item
    \(2x^2+5x-12\)
  \item
    \(3x^2-10x-8\)
  \end{enumerate}
\end{exercise}

\begin{exercise}
  Factor completely into polynomials with integer coefficients.
  
  \begin{enumerate}
  \item
    \(x^3-5x^2+6x\)
  \item
    \(4x^4-12x^2+5\)
  \item
    \(2x^3y-9x^2y^2-5xy^3\)
  \end{enumerate}
\end{exercise}

\begin{exercise}
  Each of trinomial below has a factor in the table. Match the letter on
  the left of a factor with a number on the left a trinomial to
  decipher the following quotation.
  
  % \begin{fullwidth}
  %   \colorbox{white}{
  %     \parbox{\linewidth}{
  %       \centering
\vspace*{\baselineskip}

\(\frac{\phantom{A}}{13}\)\quad \(\frac{\phantom{A}}{10~~2~~9~~15}\),\quad \(\frac{\phantom{A}}{9~~5~~14}\)\quad \(\frac{\phantom{A}}{13}\)\quad \(\frac{\phantom{A}}{4~~3~~15~~7~~2~~1}\);\\

\(\frac{\phantom{A}}{13}\)\quad \(\frac{\phantom{A}}{11~~2~~2}\),\quad \(\frac{\phantom{A}}{9~~5~~14}\)\quad \(\frac{\phantom{A}}{13}\)\quad \(\frac{\phantom{A}}{8~~5~~3~~6}\);\\

\(\frac{\phantom{A}}{13}\)\quad \(\frac{\phantom{A}}{14~~3}\),\quad \(\frac{\phantom{A}}{9~~5~~14}\)\quad \(\frac{\phantom{A}}{13}\)\quad \(\frac{\phantom{A}}{12~~5~~14~~2~~15~~11~~1~~9~~5~~14}\).
  %     }
  %   }
  % \end{fullwidth}
  
  \begin{longtable}[]{@{}
    >{\raggedright\arraybackslash}p{(\columnwidth - 12\tabcolsep) * \real{0.1429}}
    >{\raggedright\arraybackslash}p{(\columnwidth - 12\tabcolsep) * \real{0.1429}}
    >{\raggedright\arraybackslash}p{(\columnwidth - 12\tabcolsep) * \real{0.1429}}
    >{\raggedright\arraybackslash}p{(\columnwidth - 12\tabcolsep) * \real{0.1429}}
    >{\raggedright\arraybackslash}p{(\columnwidth - 12\tabcolsep) * \real{0.1429}}
    >{\raggedright\arraybackslash}p{(\columnwidth - 12\tabcolsep) * \real{0.1429}}
    >{\raggedright\arraybackslash}p{(\columnwidth - 12\tabcolsep) * \real{0.1429}}@{}}
  \toprule()
  \endhead
  \textbf{A:} \(3x-2\) & \textbf{B:} \(2x+1\) & \textbf{C:} \(x+6\) &
  \textbf{D:} \(x+7\) & \textbf{E:} \(2x-1\) & \textbf{F:} \(3x-1\) &
  \textbf{G:} \(x+10\) \\
  \textbf{H:} \(x-8\) & \textbf{I:} \(2x+9\) & \textbf{J:} \(x-1\) &
  \textbf{K:} \(x+3\) & \textbf{L:} \(2x-5\) & \textbf{M:} \(x+5\) &
  \textbf{N:} \(x-7\) \\
  \textbf{O:} \(x-13\) & \textbf{P:} \(5x-3\) & \textbf{Q:} \(4x-11\) &
  \textbf{R:} \(x-9\) & \textbf{S:} \(2x+3\) & \textbf{T:} \(x+4\) &
  \textbf{U:} \(7x+1\) \\
  \textbf{V:} \(3x+5\) & \textbf{W:} \(3x+4\) & \textbf{X:} \(8x+3\) &
  \textbf{Y:} \(x-14\) & \textbf{Z:} \(5x-6\) & & \\
  \bottomrule()
  \end{longtable}
  
  \begin{multicols}{2}
  
  \begin{enumerate}[sepno]
  \item \(x^2-2x-24\)
  \item \(6x^2+x-2\)
  \item \(x^2-16x+39\)
  \item \(6x^2+13x-5\)
  \item \(x^2-5x-14\)
  \item \(3x^2-5x-12\)
  \item \(x^2-x-110\)
  \item \(x^2-9\)  \columnbreak
  \item \(-3x^2+11x-6\)
  \item \(x^2-10x+16\)
  \item \(-2x^2+5x+12\)
  \item \(42x^2-x-1\)
  \item \(-2x^2-3x+27\)
  \item \(x^2+14x+49\)
  \item \(x^2-81\)
  \end{enumerate}
\end{multicols}
\end{exercise}

