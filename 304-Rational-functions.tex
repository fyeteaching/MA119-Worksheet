% !TeX root = main.tex

\hypertarget{rational-functions}{%
\section{Rational Functions}\label{rational-functions}}

\hypertarget{the-domain-of-a-rational-function}{%
\subsection{The Domain of a Rational
Function}\label{the-domain-of-a-rational-function}}

A \textbf{\emph{rational function}} \(f\) is defined by an equation
\(f(x)=\frac{p(x)}{q(x)}\), where \(p(x)\) and \(q(x)\) are polynomials
and the degree of \(q(x)\) is at least one. Since the denominator cannot be zero, the domain of \(f\) consists all real numbers except the
numbers such that \(q(x)=0\)

\begin{example}

Find the domain of the function \(f(x)=\frac{1}{x-1}\).

\end{example}
\vspace*{6\baselineskip}

\subsection{Practice}

\begin{exercise}

Find the domain of each function. Write in interval notation.

\begin{enumerate}
\item
  \(f(x)=\frac{x^2}{x-2}\)
\item
  \(f(x)=\frac{x}{x^2-1}\)\hfill\null
\end{enumerate}

\end{exercise}

