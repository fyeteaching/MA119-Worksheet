% !TeX root = main.tex

\hypertarget{rational-equations}{%
\section{Rational Equations}\label{rational-equations}}

\hypertarget{solving-rational-equations}{%
\subsection{Solving Rational
Equations}\label{solving-rational-equations}}

A \textbf{\emph{rational equation}} is an equation that contains a
rational expression. One way to solve rational equations is to clear all
denominators by multiplying the LCD to both sides.

\begin{example}

Solve \[
\frac{5}{x^2-9}=\frac{3}{x-3}-\frac{2}{x+3}.
\]

\end{example}
\vspace*{5\baselineskip}

Another way to solve a rational equation is to rewrite and simplify the equation into the form \(\frac{\text{A}}{B}=0\) where \(\frac{A}{B}\) is a \textbf{reduced fraction}. Then the rational equation is equivalent to the equation \(A=0\).

\begin{example}
  Solve the quation
  \[\frac{2x-3}{x - 3}-\frac{1}{x+1}=\frac{4}{x^2-2x-3}\]
\end{example}
\vspace*{5\baselineskip}

\hypertarget{literal-equations}{%
\subsection{Literal Equations}\label{literal-equations}}

A \textbf{\emph{literal equation}} is an equation involving two or more
variables. When solving a literal equation for one variable, other
variables can be viewed as constants.

\begin{example}

Solve for \(x\) from the equation \[
\frac{1}{x}+\frac{1}{y}=\frac{1}{z}.  
\]

\end{example}
\vspace*{5\baselineskip}


\subsection{Practice}

\begin{exercise}

Solve.

\begin{enumerate}
\item
  \(\dfrac1{x+1}+\dfrac1{x-1}=\dfrac4{x^2-1}\)
\item
  \(\dfrac{30}{x^2-25}=\dfrac3{x+5}+\dfrac2{x-5}\)
\end{enumerate}

\end{exercise}

\begin{exercise}

Solve.

\begin{enumerate}
\item
  \(\dfrac{2x-1}{x^2+2x-8}=\dfrac1{x-2}-\dfrac{2}{x+4}\)
\item
  \(\dfrac{3x}{x-5}=\dfrac{2x}{x+1}-\dfrac{42}{x^2-4x-5}\)
\end{enumerate}

\end{exercise}

\begin{exercise}

Solve a variable from a formula.

\begin{enumerate}
\item
  Solve for \(f\) from\quad \(\dfrac1p+\dfrac1q=\dfrac1f\).
\item
  Solve for \(x\) from\quad \(A=\dfrac{f+cx}{x}\).
\end{enumerate}

\end{exercise}

\begin{exercise}

Solve for \(x\) from the equation.

\begin{enumerate}
\item
  \(2(x+1)^{-1}+x^{-1}=2\).
\item
  \(\dfrac{a^2x +2a}{x^{-1}}=-1\).
\end{enumerate}

\end{exercise}

\begin{exercise}

David can row 3 miles per hour in still water. It takes him 90 minutes
to row 2 miles upstream and then back. How fast is the current?

\end{exercise}
\vspace*{5\baselineskip}

\begin{exercise}
The size of a A0 paper is defined to have an area of 1 square meter with
the longer dimension \(\sqrt[4]{2}\) meters. Successive paper sizes in
the series A1, A2, A3, and so forth, are defined by halving the
preceding paper size across the larger dimension. Can you find the
dimension of a A4 paper?
\end{exercise}

