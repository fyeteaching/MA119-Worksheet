% !TEX root = main.tex

\hypertarget{integer-exponents}{%
\section{Integer Exponents}\label{integer-exponents}}

%\hypertarget{dont-be-tricked}{%
%\subsection{Don't Be Tricked}\label{dont-be-tricked}}
%
%\begin{enumerate}[label=(\alph*)]
%\item
%  A pizza shop sales 12-inches pizza and 8-inches pizza at the price
%  \$12/each and \$6/each respectively. With \$12, would you like to
%  order one 12-inches and two 8-inches. Why?
%\item
%  A sheet of everyday copy paper is about 0.01 millimeter thick. Repeat
%  folding along a different side 20 times. Now, how thick do you think
%  the folded paper is?
%\end{enumerate}

\hypertarget{properties-of-exponents}{%
\subsection{Properties of Exponents}\label{properties-of-exponents}}

The $n$-th power of a real number $x$, denoted as $x^n$,  is defined as \[
x^n=\underbrace{x\cdot x \cdots x}_{n~\text{factors of}~x}.
\]

In the notation \(x^n\), \(n\) is called \textbf{\emph{the exponent}},
\(x\) is called \textbf{\emph{the base}}, and \(x^n\) is called
\textbf{\emph{the power}}.%  read as ``\(x\) raised to the \(n\)-th
%power'', ``\(x\) to the \(n\)-th power'', ``\(x\) to the \(n\)-th'',
%``\(x\) to the power of \(n\)'' or ``\(x\) to the \(n\)''.

\begin{itemize}
\item
The \textbf{product rule} \(x^m\cdot x^n=x^{m+n}.\)
\item
The \textbf{quotient rule} (for \(x\neq 0\).)
\(\frac{x^m}{x^n}= \begin{cases} x^{m-n}  & \text{if} m\ge n.\\ \frac{1}{x^{n-m}} & \text{if} m\le n. \end{cases} \)
\item
The \textbf{zero-exponent rule} (for \(x\neq 0\).) \(x^0=1.\)
\item
The \textbf{negative-exponent rule} (for \(x\neq 0\).)
\(x^{-n}=\frac{1}{x^n} \quad\text{and}\quad \frac{1}{x^{-n}}=x^n.\)
\item
The \textbf{power-to-power} rule \(\left(x^a\right)^b=x^{ab}.\)
\item
The \textbf{product-to-power} rule \((xy)^n=x^ny^n.\)
\item
The \textbf{quotient-to-power} rule (for \(y\neq 0\).)
\(\left(\frac{x}{y}\right)^n=\frac{x^n}{y^n}.\)
\end{itemize}

\begin{example}
Simplify. \textbf{Write with positive exponents.}
\begin{enumerate}
  \item $(-2x^2y^3z)(3xy^2z^2)$
  \item $\dfrac{2x^3y^2z}{6xy^3z^2}$
  \item $(-2^{10}x^3a^2p^0)^0$
  \item $(-3)^{-2}$
  \item $\left(\dfrac{(xy)^2}{xy^3}\right)^4$
  \item $\left(\dfrac{2y^{-2}z^{-5}}{4x^{-3}y^6}\right)^{-4}.$
\end{enumerate}
\end{example}
%\vspace*{5\baselineskip}

%\newpage

\subsection{Practice}

\begin{exercise}
  Simplify. \textbf{Write with positive exponents.}
  
  \begin{enumerate}[label=(\arabic*)]
  \item
    \((3a^2b^3c^2)(4abc^2)(2b^2c^3)\)
  \item
    \(\frac{4y^3z^0}{x^2y^2}\)
  \item
    \((-2)^{-3}\)
  \item
    \(\frac{-u^0v^{15}}{v^{16}}\)
  \item
    \((-2a^3b^2c^0)^3\)
  \item
    \(\frac{m^5 n^{2}}{(mn)^3}\)
  \item
    \((-3a^2x^3)^{-2}\)
  \item
    \(\left(\frac{-x^0y^3}{2wz^2}\right)^3\)
  \item
    \(\frac{3^{-2}a^{-3}b^5}{x^{-3}y^{-4}}\)
  \item
    \(\left(-x^{-1}(-y)^2\right)^3\)
  \item
    \(\left(\frac{6x^{-2}y^5}{2y^{-3}z^{-11}}\right)^{-3}\)
  \item
    \(\frac{(3 x^{2} y^{-1})^{-3}(2 x^{-3} y^{2})^{-1}}{(x^{6} y^{-5})^{-2}}\)
  \end{enumerate}
\end{exercise}

\begin{exercise}
A store has large size and small size watermelons. A large one cost \$4
and a small one \$1. Putting on the same table, a smaller watermelons
has only half the height of the larger one. Given \$4, will you buy a
large watermelon or 4 smaller ones? Why?
\end{exercise}
