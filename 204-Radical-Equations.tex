% !TeX root = main.tex

\hypertarget{radical-equations}{%
\section{Radical Equations}\label{radical-equations}}

\hypertarget{solving-radical-equations-by-taking-a-power}{%
\subsection{Solving Radical Equations by Taking a
Power}\label{solving-radical-equations-by-taking-a-power}}

The idea to solve a radical equation \(\sqrt[n]{X}=a\) is to first take
\(n\)-th power of both sides to get rid of the radical sign, that is
\(X=a^n\) and then solve the resulting equation.

\begin{example}

Solve the equation \(x-\sqrt{x+1}=1.\)

\end{example}
\vspace*{5\baselineskip}

\begin{example}

Solve the equation \(\sqrt{x-1}-\sqrt{x-6}=1.\)

\end{example}
\vspace*{5\baselineskip}

\begin{example}

Solve the equation \(-2\sqrt[3]{x-4}=6.\)

\end{example}
\vspace*{5\baselineskip}

\hypertarget{equations-involving-rational-exponents}{%
\subsection{Equations Involving Rational
Exponents}\label{equations-involving-rational-exponents}}

Equation involving rational exponents may be solved similarly. However,
one should be careful with meaning of the expression
\(\left(X^{\frac mn}\right)^{\frac nm}\). When \(m\) is even and \(n\)
is odd, \(\left(X^{\frac mn}\right)^{\frac nm}=|X|\). Otherwise,
\(\left(X^{\frac mn}\right)^{\frac nm}=X\).

\begin{example}

Solve the equation \((x+2)^{\frac12}-(x-3)^{\frac12}=1\).

\end{example}
\vspace*{5\baselineskip}

\begin{example}

Solve the equation \((x-1)^{\frac{2}{3}}=4\).

\end{example}
\vspace*{5\baselineskip}

\hypertarget{learn-from-mistakes}{%
\subsection{Learn from Mistakes}\label{learn-from-mistakes}}

\begin{example}

Can you find the mistakes made in the solution and fix it?

Solve the radical equation. \[\sqrt{x-1}+2=x\]

\textbf{Solution (incorrect)}: \[
\begin{aligned}
\sqrt{x-2}+2&=x\\
(\sqrt{x-2})^2+2^2&=x^2\\
x-2+4&=x^2\\
x+2&=x^2\\
x^2-x-2&=0\\
(x-2)(x+1)&=0\\
x-2=0 \qquad\text{or}& \qquad x+1=0\\
x=2 \qquad\text{or}&\qquad x=-1
\end{aligned}
\] 

Answer: the equation has two solutions \(x=2\) and \(x=-1\).

\end{example}
\vspace*{5\baselineskip}

\subsection{Practice}

\begin{exercise}

Solve each radical equation.

\begin{enumerate}
\item
  \(\sqrt{3x+1}=4\)
\item
  \(\sqrt{2x-1}-5=0\)
\end{enumerate}

\end{exercise}

\begin{exercise}

Solve each radical equation.

\begin{enumerate}
\item
  \(\sqrt{5x+1}=x+1\)
\item
  \(x=\sqrt{3x+7}-3\)
\end{enumerate}

\end{exercise}

\begin{exercise}

Solve each radical equation.

\begin{enumerate}
\item
  \(\sqrt{6x+7}-x=2\)
\item
  \(\sqrt{x+2}+\sqrt{x-1}=3\)
\end{enumerate}

\end{exercise}

\begin{exercise}

Solve each radical equation.

\begin{enumerate}
\item
  \(\sqrt{x+5}-\sqrt{x-3}=2\)
\item
  \(3\sqrt[3]{3x-1}=6\)
\end{enumerate}

\end{exercise}

\begin{exercise}

Solve each radical equation.

\begin{enumerate}
\item
  \((x+3)^{\frac12}=x+1\)
\item
  \(2(x-1)^{\frac12}-(x-1)^{-\frac12}=1\)
\end{enumerate}

\end{exercise}

\begin{exercise}

Solve each radical equation.

\begin{enumerate}
\item
  \((x-1)^{\frac32}=8\)
\item
  \((x+1)^{\frac23}=4\)
\end{enumerate}

\end{exercise}

