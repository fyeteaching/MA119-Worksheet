% !TeX root = main.tex

\hypertarget{radicals-and-rational-exponents}{%
\section{Radicals and Rational
Exponents}\label{radicals-and-rational-exponents}}

\hypertarget{radical-expressions}{%
\subsection{Radical Expressions}\label{radical-expressions}}

If \(b^2=a\), then we say that \(b\) is a \textbf{\emph{square root}} of
\(a\). We denote the positive square root of \(a\) as \(\sqrt{a}\),
called the \textbf{\emph{principal square root}}.

For any real number \(a\), the expression \(\sqrt{a^2}\) can be
simplified as \[
\sqrt{a^2}=|a|.
\]

If \(b^3=a\), then we say that \(b\) is a \textbf{\emph{cube root}} of
\(a\). The cube root of a real number \(a\) is denoted by
\(\sqrt[3]{a}\).

For any real number \(a\), the expression \(\sqrt[3]{a^3}\) can be
simplified as \[
\sqrt[3]{a^3}=a.
\]

In general, if \(b^n=a\), then we say that \(b\) is an
\textbf{\emph{\(n\)-th root}} of \(a\). If \(n\) is \textbf{even}, the
\textbf{positive} \(n\)-th root of \(a\), called the
\textbf{\emph{principal \(n\)-th root}}, is denoted by \(\sqrt[n]{a}\).
If \(n\) is odd, the \(n\)-the root \(\sqrt[n]{a}\) of \(a\) has the
same sign with \(a\).

In \(\sqrt[n]{a}\), the symbol \(\sqrt{\phantom{a}}\) is called the
\textbf{\emph{radical sign}}, \(a\) is called the
\textbf{\emph{radicand}}, and \(n\) is called the \textbf{\emph{index}}.

If \(n\) is even, then the \(n\)-th root of a negative number is not a
real number.

For any real number \(a\), the expression \(\sqrt[n]{a^n}\) can be
simplified as

\begin{enumerate}[sepno]
\item
  \(\sqrt[n]{a^n}=|a|\) if \(n\) is even.
\item
  \(\sqrt[n]{a^n}=a\) if \(n\) is odd.
\end{enumerate}

A radical is simplified if the radicand has no perfect power factors against the radical.

\begin{example}
  Simplify the radical expression using the definition.
  
  \begin{enumerate}
  \item
    \(\sqrt{4(y-1)^2}\)
  \item
    \(\sqrt[3]{-8x^3y^6}\)
  \item
    \(\sqrt{4(y-1)^2}=\sqrt{[2(y-1)]^2}=2|y-1|\).
  \item
    \(\sqrt[3]{-8x^3y^6}=\sqrt[3]{(-2xy^2)^2}=-2xy^2\).
  \end{enumerate}
\end{example}

\hypertarget{rational-exponents}{%
\subsection{Rational Exponents}\label{rational-exponents}}

If \(\sqrt[n]{a}\) is a real number, then we define \(a^{\frac mn}\) as
\[
a^{\frac mn}=\sqrt[n]{a^m}=(\sqrt[n]{a})^m.
\]

Rational exponents have the same properties as integral exponents:

\begin{enumerate}[sepno]
\item
  \(a^m\cdot a^n=a^{m+n}\)
\item
    \(\dfrac{a^m}{a^n}=a^{m-n}\)
\item
  \(a^{-\frac mn}=\dfrac{~1~}{~a^{\frac mn}~}\)\\
\item
  \((a^m)^n=a^{mn}\)
\item
  \((ab)^m=a^m\cdot b^m\)
\item
  \(\left(\dfrac ab\right)^m=\dfrac{a^m}{b^m}\)
\end{enumerate}

\begin{example}
  Simplify the radical expression or the expression with rational
  exponents. \textbf{Write in radical notation}.
  
  \begin{enumerate}
  \item
    \(\sqrt{x}\sqrt[3]{x^2}\)
  \item
    \(\sqrt[3]{\sqrt{x^3}}\)
  \item
    \(\left(\dfrac{x^{\frac12}}{x^{-\frac56}}\right )^{\frac14}\)
  \item
    \(\sqrt{\dfrac{x^{-\frac12}y^2}{x^{\frac32}}}\)
  \end{enumerate}
\end{example}

In general, rewriting radical in rational exponents helps simplify
calculations.

\hypertarget{product-and-quotient-rules-for-radicals}{%
\subsection{Product and Quotient Rules for
Radicals}\label{product-and-quotient-rules-for-radicals}}

If \(\sqrt[n]{a}\) and \(\sqrt[n]{b}\) are real numbers, then
\[{\sqrt[n]a}{\sqrt[n]b}=\sqrt[n]{ab}.\]

If \(\sqrt[n]a\) and \(\sqrt[n]b\) are real numbers and \(b\neq 0\),
then \[\dfrac{\sqrt[n]a}{\sqrt[n]b}=\sqrt[n]{\dfrac ab}.\]

\begin{example}
  Simplify the expression.
  
  \begin{enumerate}
  \item
    \(\sqrt[4]{8xy^4}\sqrt[4]{2x^7y}\).
  \item
    \(\dfrac{\sqrt[5]{96x^9y^3}}{\sqrt[5]{3x^{-1}y}}\).
  \end{enumerate}
\end{example}

\hypertarget{combining-like-radicals}{%
\subsection{Combining Like Radicals}\label{combining-like-radicals}}

Two radicals are called \textbf{\emph{like radicals}} if they have the
same index and the same radicand. We add or subtract like radicals by
combining their coefficients.

\begin{example}
  Simplify the expression.
  
  \[
  \sqrt{8x^3}-\sqrt{(-2)^2 x^4}+\sqrt{2x^5}.
  \]
\end{example}
\vspace*{4\baselineskip}

\hypertarget{multiplying-radicals}{%
\subsection{Multiplying Radicals}\label{multiplying-radicals}}

Multiplying radical expressions with many terms is similar to that
multiplying polynomials with many terms.

\begin{example}
  Simplify the expression. \[
  (\sqrt{2x}+2\sqrt{x})(\sqrt{2x}-3\sqrt{x}).
  \]
\end{example}
\vspace*{4\baselineskip}

\hypertarget{rationalizing-denominators}{%
\subsection{Rationalizing
Denominators}\label{rationalizing-denominators}}

Rationalizing denominator means rewriting a radical expression into an
equivalent expression in which the denominator no longer contains
radicals.

\begin{example}
  Rationalize the denominator.
  
  \begin{enumerate}
  \item
    \(\dfrac{1}{2\sqrt{x^3}}\)
  \item
    \(\dfrac{\sqrt{x}+\sqrt{y}}{\sqrt{x}-\sqrt{y}}\)
  \end{enumerate}
\end{example}

\hypertarget{complex-numbers}{%
\subsection{Complex Numbers}\label{complex-numbers}}

The imaginary unit \(\ii\) is defined as \(\ii=\sqrt{-1}\). Hence
\(\ii^2=-1\).

If \(b\) is a positive number, then \(\sqrt{-b}=\ii\sqrt{b}\).

Let \(a\) and \(b\) are two real numbers. We define a complex number by
the expression \(a+b \ii\). The number \$a \$ is called the real part
and the number \(b\) is called the imaginary part. If \(b=0\), then the
complex number \(a+b\ii=a\) is just the real number. If \(b\neq 0\),
then we call the complex number \(a+b\ii\) an imaginary number. If
\(a=0\) and \(b\neq 0\), then the complex number \(a+b\ii=b\ii\) is
called a purely imaginary number.

\textbf{Adding, subtracting, multiplying, dividing or simplifying
complex numbers are similar to those for radical expressions. In
particular, adding and subtracting become similar to combining like
terms.}

\begin{example}
  Simplify and write your answer in the form \(a+b\ii\), where \(a\) and
  \(b\) are real numbers and \(\ii\) is the imaginary unit.
  
  \begin{enumerate}
  \item
    \(\sqrt{-3}\sqrt{-4}\)
  \item
    \((4\ii-3)(-2+\ii)\)
  \item
    \(\dfrac{-2+5\ii}{\ii}\)
  \item
    \(\dfrac{1}{1-2\ii}\)
  \item
    \(\ii^{2018}\)
  \end{enumerate}
\end{example}

\begin{example}
  Evaluate the express \(z^2+\dfrac{z-1}{z+1}\) for \(z=1+\ii\). Write your
  answer in the form \(a+b\ii\).
\end{example}
\vspace*{4\baselineskip}

\subsection{Practice}

\begin{exercise}
  Evaluate the square root. If the square root is not a real number, state
  so.
  
  \begin{enumerate}
  \item
    \(-\sqrt{\dfrac{4}{25}}\)
  \item
    \(\sqrt{49}-\sqrt{9}\)
  % \item
  %   \(-\sqrt{-1}\)\null
  \end{enumerate}
\end{exercise}

\begin{exercise}
  Simplify the radical expression.
  
  \begin{enumerate}
  \item
    \(\sqrt{(-7x^2)^2}\)
  \item
    \(\sqrt{(x+2)^2}\)
  % \item
  %   \(\sqrt{25x^2y^6}\)
  \end{enumerate}
\end{exercise}

\begin{exercise}
  Simplify the radical expression.
  
  \begin{enumerate}
  \item
    \(\sqrt[3]{-27x^3}\)
  % \item
  %   \(\sqrt[4]{16x^8}\)
  \item
    \(\sqrt[5]{(2x-1)^5}\)
  \end{enumerate}
\end{exercise}

\begin{exercise}
  Simplify the radical expression. Assume all variables are positive.
  
  \begin{enumerate}
  \item
    \(\sqrt{50}\)
  \item
    \(\sqrt[3]{-8x^2y^3}\)
  % \item
  %   \(\sqrt[5]{32x^{12}y^2z^8}\)
  \end{enumerate}
\end{exercise}

\begin{exercise}
  Write the radical expression with rational exponents.
  
  \begin{enumerate}
  \item
    \(\sqrt[3]{(2x)^5}\)
  \item
    \((\sqrt[5]{3xy})^7\)
  \item
    \(\sqrt[4]{(x^2+3)^3}\)
  \end{enumerate}
\end{exercise}

\begin{exercise}
  Write in radical notation and simplify.
  
  \begin{enumerate}
  \item
    \(4^{\frac32}\)
  \item
    \(-81^{\frac 34}\)
  \item
    \(\left(\frac{27}{8}\right)^{-\frac{2}{3}}\)
  \end{enumerate}
\end{exercise}

\begin{exercise}
  Simplify the expression. Write with radical notations. Assume all
  variables represent nonnegative numbers.
  
  \begin{enumerate}
  \item
    \(\dfrac{12x^{\frac12}}{4x^{\frac23}}\)
  \item
    \((x^{-\frac35}y^{\frac12})^{\frac13}\)
  \item
    \(\left(\dfrac{x^{\frac12}}{x^{-\frac13}}\right)^4\)
  \end{enumerate}
\end{exercise}

\begin{exercise}
  Simplify the expression. Write in radical notation. Assume \(x\) is
  nonnegative.
  
  \begin{enumerate}
  \item
    \(\dfrac{\sqrt{x}}{\sqrt[3]{x}}\)
  \item
    \(\sqrt{\sqrt[3]{x}}\)
  \item
    \(\sqrt{x}\sqrt[3]{x}\)
  \end{enumerate}
\end{exercise}

\begin{exercise}
  Simplify the expression. Write in radical notation. Assume \(x\) is
  nonnegative.
  
  \begin{enumerate}
  \item
    \(\sqrt[5]{32x^{\frac13}}\)
  \item
    \(\left(\dfrac{\sqrt[4]{9x}}{3}\right)^{-2}\)
  % \item
  %   \(\sqrt{\dfrac{1}{\sqrt[3]{x^{-2}}}}\)
  \end{enumerate}
\end{exercise}

\begin{exercise}
  Simplify the expression. Write in radical notation. Assume all variables
  are nonnegative.
  
  \begin{enumerate}
  \item
    \(\left(\dfrac{8a^{-\frac{5}{2}}b}{a^{\frac12}b^{-5}}\right)^{-\frac23}\)
  \item
    \(\left(\dfrac{y^{-\frac{1}{3}}}{\sqrt[3]{x^{2}}}\right)^{-3}\)
  \item
    \(\sqrt[3]{(-x)^{-2}}\sqrt{x^3}\)
  \end{enumerate}
\end{exercise}

\begin{exercise}
  Multiply and simplify.
  
  \begin{enumerate}
  \item
    \(\sqrt[3]{4}\sqrt[3]{5}\)
  \item
    \(\sqrt{|x+7|}\sqrt{|x-7|}\)
  \item
    \(\sqrt[3]{(x-y)^{\frac52}}\sqrt[3]{(x-y)^{\frac72}}\)
  \end{enumerate}
\end{exercise}

\begin{exercise}
  Simplify the radical expression. Assume all variables are positive.
  
  \begin{enumerate}
  \item
    \(\sqrt{20xy}\cdot\sqrt{4xy^2}\)
  % \item
  %   \(\sqrt[3]{16}\cdot5\sqrt[3]{2}\)
  \item
    \(\sqrt[5]{8x^4y^3z^3}\cdot\sqrt[5]{8xy^4z^8}\)
  \end{enumerate}
\end{exercise}

\begin{exercise}
  Divide. Assume all variables are positive. Answers must be simplified.
  
  \begin{enumerate}
  \item
    \(\sqrt{\dfrac{9x^3}{y^8}}\)
  % \item
  %   \(\sqrt[3]{\dfrac{32x^4}{x}}\)
  % \item
  %   \(\dfrac{\sqrt{40x^5}}{\sqrt{2x}}\)
  \item
    \(\dfrac{\sqrt[3]{24a^6b^4}}{\sqrt[3]{3b}}\)
  \end{enumerate}
\end{exercise}

\begin{exercise}
  Add or subtract, and simplify. Assume all variables are positive.
  
  \begin{enumerate}
  % \item
  %   \(5\sqrt6+3\sqrt6\)
  \item
    \(4\sqrt{20}-2\sqrt5\)
  \item
    \(3\sqrt{32x^2}+5x\sqrt{8}\)
  \end{enumerate}
\end{exercise}

\begin{exercise}
  Add or subtract, and simplify. Assume all variables are positive
  
  \begin{enumerate}
  \item
    \(7\sqrt{4x^2}+2\sqrt{25x}-\sqrt{16x}\)
  \item
    \(5\sqrt[3]{x^2y}+\sqrt[3]{27x^5y^4}\)
  \item
    \(3\sqrt{9y^3}-3y\sqrt{16y}+\sqrt{25y^3}\)
  \end{enumerate}
\end{exercise}

\begin{exercise}
  Multiply and simplify. Assume all variables are positive.
  
  \begin{enumerate}
  \item
    \(\sqrt2(3\sqrt3-2\sqrt2)\)
  \item
    \((\sqrt5+\sqrt7)(3\sqrt5-2\sqrt7)\)
  \item
    \((\sqrt3+\sqrt2)^2\)
  % \item
  %   \((\sqrt6-\sqrt5)(\sqrt6+\sqrt5)\)
  \item
    \((\sqrt{x+1}-1)(\sqrt{x+1}+1)\)
  \item
    \((2\sqrt[3]x+6)(\sqrt[3]x+1)\)
  \end{enumerate}
\end{exercise}

\begin{exercise}
  Simplify the radical expression and rationalize the denominator. Assume
  all variables are positive.
  
  \begin{enumerate}
  % \item
  %   \(\sqrt[3]{\dfrac2{25}}\)
  \item
    \(\sqrt{\dfrac{2x}{7y}}\)
  \item
    \(\dfrac{\sqrt[3]{x}}{\sqrt[3]{3y^2}}\)
  % \item
  %   \(\dfrac{3x}{\sqrt[4]{x^3y}}\)
  \end{enumerate}
\end{exercise}

\begin{exercise}
  Simplify the radical expression and rationalize the denominator. Assume
  all variables are positive.
  
  \begin{enumerate}
  \item
    \(\dfrac{6\sqrt3}{\sqrt3-1}\)
  \item
    \(\dfrac{\sqrt5-\sqrt3}{\sqrt5+\sqrt3}\)
  \item
    \(\dfrac{3+\sqrt2}{2+\sqrt3}\)
  \item
    \(\dfrac{2\sqrt{x}}{\sqrt x- \sqrt y}\)
  \end{enumerate}
\end{exercise}

\begin{exercise}
  Simplify and rationalize the denominator. Assume all variables are
  positive.
  
  \begin{enumerate}
  \item
    \(\dfrac{\sqrt{x}}{\sqrt x-1}+\dfrac{1}{\sqrt{x}+1}\)
  \item
    \(\dfrac{\sqrt{x}+1}{\sqrt x}-\dfrac{1}{\sqrt{x}-1}\)
  \end{enumerate}
\end{exercise}

\begin{exercise}
  Add, subtract, multiply complex numbers and write your answer in the
  form \(a+b\ii\).
  
  \begin{enumerate}
  \item
    \(\sqrt{-2}\cdot\sqrt{-3}\)
  \item
    \(\sqrt{2}\cdot\sqrt{-8}\)
  \item
    \((5-2\ii)+(3+3\ii)\)
  \item
    \((2+6\ii)-(12-4\ii)\)
  \end{enumerate}
\end{exercise}

\begin{exercise}
  Add, subtract, multiply complex numbers and write your answer in the
  form \(a+b\ii\).
  
  \begin{enumerate}
  \item
    \((3+\ii)(4+5\ii)\)
  \item
    \((7-2\ii)(-3+6\ii)\)
  \item
    \((3-x\sqrt{-1})(3+x\sqrt{-1})\)
  \item
    \((2+3\ii)^2\)
  \end{enumerate}
\end{exercise}

\begin{exercise}
  Divide the complex number and write your answer in the form \(a+b\ii\).
  
  \begin{enumerate}
  \item
    \(\dfrac{2\ii}{1+\ii}\)
  \item
    \(\dfrac{5-2\ii}{3+2\ii}\)
  \item
    \(\dfrac{2+3\ii}{3-\ii}\)
  \item
    \(\dfrac{4+7\ii}{-3\ii}\)
  \end{enumerate}
\end{exercise}

\begin{exercise}
  Simplify the expression.
  
  \begin{enumerate}
  \item
    \((-\ii)^{8}\)
  % \item
  %   \(\ii^{15}\)
  \item
    \(\ii^{2021}\)
  \item
    \(\dfrac1{\ii^{2022}}\)
  \end{enumerate}
\end{exercise}

\begin{exercise}
  Evaluate the function polynomial \(2x^2-3x+5\) for \(x=1-\ii\). Write
  your answer in the form \(a+b\ii\).
\end{exercise}
\vspace*{5\baselineskip}

\begin{exercise}
  Evaluate the polynomial \(\ii x^2-x+\dfrac{2}{x-1}\) for \(x=\ii-1\).
  Write your answer in the form \(a+b\ii\).
\end{exercise}

